\section{Utilisation de la fonction \texttt{kron}}
\label{app:kron}

\lstset{language=MATLAB}

Afin de comprendre l'utilité et l'efficacité de la fonction \texttt{kron},
voici son application pour la contrainte
\[ 
  \Delta\xstock + \texttts{demande}(s) = \xn + \xsup 
  + \xretard + \xsst - \myX{s-1}{retard} \qquad \forall s
\]
pour une planification sur deux semaines ($\texttt{d.T}=2$).

Voici le code correspondant
\begin{lstlisting}
  dSeg   = [1 1 -1 1 1];
  dpSeg  = [0 0 1 -1 0];
  Aeq = [kron([eye(d.T),zeros(d.T,1)],dpSeg) ...
       + kron([zeros(d.T,1),eye(d.T)],dSeg)];
  beq = [d.demande'];   
\end{lstlisting}

L'instruction \lstinline{kron(A,B)} effectue l'opération suivante
\[ 
  \texttts{kron}
    \left(
      \begin{pmatrix} a_{1,1} & a_{1,2} \\ a_{2,1} & a_{2,2} \end{pmatrix}, 
      \texttts{B}
    \right)
  =
  \begin{pmatrix} a_{1,1} \cdot\texttts{B} & a_{1,2} \cdot\texttts{B}
    \\ a_{2,1}\cdot\texttts{B} & a_{2,2} \cdot\texttts{B}
  \end{pmatrix}
\]

Nous avons donc tout d'abord la matrice créée à la ligne $3$
\[
    \left(
    \begin{array}{*{15}c}
      0 & 0 & 1 & -1 & 0 & 0 & 0 &  0 & 0 & 0 & 0 & 0 & 0 & 0 & 0 \\
      0 & 0 & 0 &  0 & 0 & 0 & 0 & -1 & 1 & 0 & 0 & 0 & 0 & 0 & 0 
    \end{array}
    \right),
\]

ensuite celle créée à la ligne $4$
\[
    \left(
    \begin{array}{*{15}c}
      0 & 0 & 0 & 0 & 0 & 1 & 1 & -1 & 1 & 1 & 0 & 0 & 0 & 0 & 0 \\
      0 & 0 & 0 & 0 & 0 & 0 & 0 &  0 & 0 & 0 & 1 & 1 & -1 & 1 & 1 
    \end{array}
    \right).
\]


Rappelons la forme de notre vecteur $x_s$
\[ x_s = 
  \begin{pmatrix} 
    \xn &\xsup &\xstock &\xretard &\xsst
  \end{pmatrix}^T.
\]

Cela donne comme voulu $A_{eq} \, x = b_{eq}$
\[
    \left(
    \begin{array}{*{15}c}
      0 & 0 & 1 & -1 & 0 & 1 & 1 & -1 & 1 & 1 & 0 & 0 &  0 & 0 & 0 \\
      0 & 0 & 0 &  0 & 0 & 0 & 0 & -1 & 1 & 0 & 1 & 1 & -1 & 1 & 1 
    \end{array}
    \right)
    \left(
    \begin{array}{*{1}l}
      \myX{0}{n}\\ \myX{0}{sup}\\ \myX{0}{stock}\\ \myX{0}{retard}\\ \myX{0}{sst}\\
      \myX{1}{n}\\ \myX{1}{sup}\\ \myX{1}{stock}\\ \myX{1}{retard}\\ \myX{1}{sst}\\
      \myX{2}{n}\\ \myX{2}{sup}\\ \myX{2}{stock}\\ \myX{2}{retard}\\ \myX{2}{sst}
    \end{array}
    \right)
    = 
    \left(
    \begin{array}{*{1}l}
      \texttts{demande}(1) \\ \texttts{demande}(2)
    \end{array}
    \right)
    \text.
\]

Cette méthode s'avère simple et efficace puisque ces $5$ lignes de code suffisent 
à remplir notre matrice peu importe le nombre de semaines.
