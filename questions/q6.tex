\question %6
\emph{Décrivez (sans l'implémenter) l'adaptation qu'il serait nécessaire à
apporter au modèle si le coût de l'heure supplémentaire pris en compte était
variable. Plus concrètement, considérez qu'après la première heure
supplémentaire (facturée au coût horaire \texttt{cout\textunderscore
heure\textunderscore sup} standard), chaque heure supplémentaire
(éventuellement) est facturée à un coût horaire supérieur de $5 \%$ 
à celui de l'heure supplémentaire précédente. 
Est-il toujours possible de formuler (ou reformuler) 
le problème sous forme linéaire ? Expliquez. 
Et que se passerait-il si le coût horaire des supplémentaires \emph{diminuait}
lorsque le nombre d'heure prestées augmente ? Justifiez.}

Si le coût des heures supplémentaires n'est plus constant mais augmente au fur et à mesure de leur utilisation, 
le coût de celles-ci dépenderait directement du nombre de smartphones 
produits de cette manière.

En effet, on peut réécrire notre fonction objectif comme
\[
  \mbox{minimiser } 
  \sum_{s=1}^{T} 
  c_m\, \xn + c_m \, \xsup + \mathcal{C}_{hs} (\xsup)
  + c_s\, \xstock + c_r\, \xretard + c_{sst}\, \xsst
\]
avec
\[
  \mathcal{C}_{hs} (\xsup) = \texttts{cout\_heure\_sup} \cdot
  \sum_{i=0}^{\alpha} \binom{\alpha}{i} (0.05)^{i} 
\]
Par soucis de lisibilité, on a posé
\[ \alpha = \xsup \, d_{a,h} - 1 \]
$\alpha$ correspond donc au nombre d'heures supplémentaires qui auront
un cout différent de \texttt{cout\_heure\_sup}.

On remarque immédiatement que la fonction $\mathcal{C}_{hs}$
est \emph{non-linéaire} en $\xsup$ puisque 
\[ \binom{\alpha}{i} =\frac{\alpha !}{i!\,(\alpha-i)!} \]

De même, si le coût des heures supplémentaires \emph{diminuait} 
avec un taux de $0.05$,
on aurait
\[
  c_{hs} (\xsup) = \texttts{cout\_heure\_sup} \cdot
  \sum_{i=0}^{\alpha} (-1)^i(0.05)^{i} \,   \binom{\alpha}{i}
\]
