\question %10
\emph{Critiquez les modèles proposés dans ce projet. Sont-ils réalistes ?
Des approximations ont-elles été faites et, si oui sont-elles justifiées ?
Quelles améliorations pourriez-vous proposer (sans rentrer dans les détails),
avec quel impact potentiel sur la résolution du problème.}

%Grande différence entre les deux modèles => instabilité
%Statismisation :)
%Cout de retard, difficile à évauer en pratique (dégradation de l'image)

Dans notre modèle, nous avons fait l'approximation que les smartphones étaient soit produits au salaire normal, soit au salaire des heures supplémentaires. Or, on pourrait théoriquement produire un smartphone en partie pendant les heures normales et en partie pendant les heures supplémentaires. Cette approximation a été réalisée afin de simplifier la modélisation. Les conséquences de cette approximation restent néanmoins très réduites puisqu'elles ne concernent qu'un seul smartphone tout au plus. La différence par rapport au coût optimal serait donc très faible, pour ne pas dire négligeable.

Il est intéressant de souligner la différence relativement grande que nous avons obtenue pour les deux types de modèles. Simplifier le modèle peut conduire à des différences importantes. Or, en pratique, on est toujours obligé de simplifier, c'est d'ailleurs le propre d'un modèle. Nous n'obtiendrons probablement jamais la solution qui minimiserait effectivement le coût mais ces modèles nous permettent de nous faire une idée de cette solution, de l'approcher et de baser nos décisions sur des considérations rationnelles. 

Un exemple de simplification utilisée dans notre modèle est la valeur fixée a priori du coût de retard. En pratique, il est difficile d'évaluer un tel coût. Il y aura peut-être des coûts directs comme une petite réduction sur le prix fait au client si son produit lui arrive en retard, mais il y aura surtout des coûts indirects liés notamment à la dégradation de l'image de l'entreprise. Nous tentons donc d'appliquer un modèle mathématique à l'économie qui, au contraire des mathématiques, est plutôt une science humaine que scientifique basée notamment sur le risque : est il profitable de prendre le risque de mécontenter des clients pour économiser quelques euros ?

En outre, en pratique, il est très rare pour une entreprise de connaître parfaitement son carnet de demande plusieurs mois à l'avance. Il faudra là encore avoir recours à une approximation et la solution mathématique optimale n'en sera que plus éloignée de la réalité.

Dans un problème comme celui-ci, il est aisé de trouver de nouveaux paramètres dont on pourrait tenir compte dans notre modèle afin de s'approcher encore un peu plus de la solution réellement optimale. On pense notamment à tenir compte d'un éventuel coût de production différent en fonction de l'usine dans laquelle on produirait l'objet. On pourrait également tenir compte du fait qu'il existe un coût fixe dans une usine et donc que ne pas l'utiliser au maximum de sa capacité constitue un manque à gagner. Ou encore, que plus le nombre de smartphones produits en interne est important, plus un certain nombre de coûts internes à l'entreprise viendraient se rajouter comme la probabilité de produire un téléphone défectueux ou la probabilité d'accident sur le lieu de travail. Cependant, les exemples cités ici n'auraient sans doute qu'un impact très limité sur la solution dans une petite production.