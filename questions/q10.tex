\question %10
\emph{Critiquez les modèles proposés dans ce projet. Sont-ils réalistes ?
Des approximations ont-elles été faites et, si oui sont-elles justifiées ?
Quelles améliorations pourriez-vous proposer (sans rentrer dans les détails),
avec quel impact potentiel sur la résolution du problème.}

%Grande différence entre les deux modèles => instabilité
%Statismisation :)
%Cout de retard, difficile à évauer en pratique (dégradation de l'image)

Il est intéressant de souligner la différence relativement grande que nous avons obtenue pour les deux types de modèles. Simplifier le modèle peut conduire à des différences importantes. Or, en pratique, on est toujours obligé de simplifier, c'est d'ailleurs le propre d'un modèle. Nous n'obtiendrons probablement jamais la solution qui minimiserait effectivement le coût mais ces modèles nous permettent de nous faire une idée de cette solution, de l'approcher et de baser nos décisions sur des considérations rationnelles. 

Un exemple de simplification utilisée dans notre modèle est la valeur fixée à priori du coût de retard. En pratique, il est difficile d'évaluer un tel coût. Il y aura peut-être des coûts directs comme une petite réduction sur le prix fait au client si son produit lui arrive en retard, mais il y aura surtout des coûts indirects liés notamment à la dégradation de l'image de l'entreprise. Nous tentons donc d'appliquer un modèle mathématique à l'économie qui, au contraire des mathématiques, est plutôt une science humaine que scientifique basée notamment sur le risque : est il profitable de prendre le risque de mécontenter des clients pour économiser quelques euros ?