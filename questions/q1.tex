\question %1
\emph{Donnez une formulation linéaire (continue, sans variables entières)
du problème de la planification de la ligne d'assemblage à personnel constant.
Décrivez successivement variables, contraintes et fonction objectif.
A ce stade, le fait de ne pas imposer l'intégralité des variables
vous parait-il problématique ?}

\subsection*{Variables}
Le tableau~\ref{tab:variablesQuestion1} contient les différentes variables $x_{s,\lambda}$
qui correspondent au nombre de smartphones pour chaque semaine $s$
avec la caractéristique $\lambda$.

\begin{table}[h]
  \begin{center}
  \begin{tabular}{|l|l|}
    \hline
    Variable & Caractéristiques des smartphones \\
    \hline
    \hline
    $\xn$ & Produits au \emph{salaire normal}. \\
    \hline
    $\xsup$ & Produits pendant les \emph{heures supplémentaires}. \\
    \hline
    $\xstock$ & Conservés en \emph{stock}. \\
    \hline
    $\xretard$ & Vendus une semaine en \emph{retard}. \\
    \hline
    $\xsst$ & Sous-traités. \\
    \hline
  \end{tabular}
  \caption{Variables de la modélisation de la ligne d'assemblage simple.}
  \label{tab:variablesQuestion1}
  \end{center}
\end{table}

\subsection*{Contraintes}
Voici les contraintes du problème de la planification
de la ligne d’assemblage à personnel constant.
On pose que $\Delta x_{s,\lambda} = x_{s,\lambda} - x_{s-1,\lambda}$.
Voici les contraintes du problème de la planification 
de la ligne d’assemblage à personnel constant, $s$ étant un naturel allant de $1$ à $T$.
% J'ai retiré
%  \myX{0}{n}&= 0 \\
% \myX{0}{sup}&= 0 \\
% \myX{0}{sst}&= 0 \\
% Puisque s va de 1 à T
\begin{align*}
  \Delta\xstock + \texttts{demande}(s) &= \xn + \xsup
   + \xsst + \Delta\xretard &\forall s \\
  \myX{T}{stock} &= \texttts{stock\_initial} \\
  \myX{T}{retard}&= 0 \\
  \myX{s-1}{retard} + \Delta\xstock &\leq \xn + \xsup + \xsst \text{\footnotemark} &\forall s \\
  \xn &\leq 35\cdot \texttts{nb\_ouvriers}/ d_{a,h}
  &\forall s \\
  \xsup &\leq \texttts{nb\_max\_heure\_sup}\cdot\texttts{nb\_ouvriers}/ d_{a,h}
  &\forall s \\
  \xsst &\leq \texttts{nb\_max\_sous\_traitant} &\forall s \\
  x_{s,\lambda} &\geq 0 &\forall s,\lambda \\
\end{align*}
\footnotetext{Pour limiter les retards à une seule semaine.}

\subsection*{Fonction objectif}
\[ \text{minimiser } \text{coût}_\text{tot} = \sum_{s=1}^{T} \text{coût}(s) \]

où coût$(s)$ est le coût pour la semaine $s$ et vaut
\[
  c_m\, \xn + (c_m + d_{a,h} \, c_{hs})\, \xsup
  + c_s\, \xstock + c_r\, \xretard + c_{sst}\, \xsst.
\]

Le tableau~\ref{tab:constantesQuestion1} contient les abréviations
des constantes utilisées.
\begin{table}[h]
  \begin{center}
  \begin{tabular}{|l|l|}
    \hline
    Paramètre & Constante représentée \\
    \hline
    \hline
    $c_m$ & \texttt{cout\_materiaux} \\
    \hline
    $c_{hs}$ & \texttt{cout\_heure\_sup} \\
    \hline
    $c_s$ & \texttt{cout\_stockage} \\
    \hline
    $c_r$ & \texttt{cout\_retard} \\
    \hline
    $c_{sst}$ & \texttt{cout\_sous\_traitant} \\
    \hline
    $d_{a,h}$ & \texttt{duree\_assemblage}/60, [heures] \\
    \hline
  \end{tabular}
  \caption{Constantes de la modélisation de la ligne d'assemblage simple.}
  \label{tab:constantesQuestion1}
  \end{center}
\end{table}

On observe tout de suite que $1/d_{a,h}$ représente le nombre de smartphones assemblables par heure pour un ouvrier.

On remarquera que le terme associé au coût des heures normales des ouvriers
a été omis dans la fonction objectif puisqu'il n'est pas nécessaire.
En effet, les ouvriers ne sont pas payés en fonction du nombre
d'heures qu'ils travaillent réellement mais bien à la semaine et
donc pour un nombre d'heures constant.
Ceci correspond au terme $35\cdot\texttt{cout\_horaire}\cdot
\texttt{nb\_ouvriers}$ qu'il est impossible de minimiser.
Cela n'aura donc pas d'impact sur notre solution,
il faudra seulement modifier le coût total de la manière suivante
\[ \text{coût}_\text{tot} \leftarrow \text{coût}_\text{tot} + T \cdot 35 \cdot\texttt{cout\_horaire}\cdot
\texttt{nb\_ouvriers} \]

A ce stade, le fait de ne pas imposer l'intégralité des variables
parait problématique dans le sens où les solutions ne sont pas 
garanties d'être entières. Ce qui n'est pas envisageable 
vu que celles-ci représentent des quantités de smartphones.
Par exemple, il est possible que
$\xn$ ne soit pas entier si $35\cdot \texttts{nb\_ouvriers}/ d_{a,h}$ ne l'est pas.
