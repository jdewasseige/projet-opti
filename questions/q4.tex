\question %4
\emph{Décrivez une procédure permettant, avec le moins de nouveaux calculs
possibles, d'évaluer les conséquences sur la fonction objectif d’une petite
variation de la demande prévue. Plus précisément, analysez l'effet du
remplacement du vecteur \texttt{demande} 
par le vecteur \texttt{demande + epsilon * delta\textunderscore demande} 
où \texttt{delta\textunderscore demande} est un vecteur de perturbation 
sur la demande, 
et \texttt{epsilon} est un paramètre scalaire dont la valeur est faible.}

Le dual du problème nous permet d'évaluer assez simplement
les conséquences sur la fonction objectif
d'une petite variation de la demande prévue.

En effet, notre problème peut être simplifié sous la forme
\begin{align*} 
	\text{minimiser } c^T x \\
	a_i^T x &= b_i  & i = 1,...,T,...,T+7 \\
  a_i^T x &\leq b_i & i = T+8,...,end \\
	x_j &\geq 0 & \forall j
\end{align*}
On obtient ensuite sa forme duale
\begin{align*} 
	\text{maximiser } b^T y \\
	y_i &\text{ libre} & i = 1,...,T,...,T+7 \\
  y_i &\leq 0 & i = T+8,...,end \\
  A_j^T y &\leq c_j & \forall j
\end{align*}

%Une partie du vecteur $b$ (et plus précisement les $T$ premiers éléments)
%correspond à la demande.
On remarque qu'une perturbation des contraintes dans le problème primal
correspond à une perturbation de la fonction objectif dans le problème dual.
Il nous suffit donc simplement de calculer une solution optimale du dual $y^{*}$
puis d'effectuer le produit scalaire
\[ \Delta z^{*} = (\Delta b)^T y^{*} \]
pour chaque perturbation $\Delta b$ pour conna\^itre l'impact $\Delta z^{*}$
sur le cout.

Ce résultat n'est valable que pour des petites variations de b.
En effet, pour l'obtenir, il faut supposer que l'on connaît un sommet optimal admissible.
Or, en changeant le problème, il n'est pas garanti que ce sommet reste optimal et admissible. 
Plus la perturbation est importante, plus il y a de chances que notre sommet optimal change.
