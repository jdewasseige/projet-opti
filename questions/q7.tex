\question %7
\emph{Donnez à présent une formulation linéaire (continue, sans variables
entières) du problème de la planification de la ligne d'assemblage
incluant le gestion du personnel, en vous basant sur le modèle déjà 
construit à la Question 1. Décrivez successivement variables, 
contraintes et fonction objectif.}

Le nombre d'ouvriers n'étant plus constant, nous allons rajouter trois variables
par semaine pour modéliser le problème :
\begin{itemize}
  \item[$\diamond$] $\nemb$ le nombre d'ouvriers embauchés au début de la semaine $s$,
  \item[$\diamond$] $\nlic$ le nombre d'ouvriers licienciés au début de la semaine $s$,
  \item[$\diamond$] $\nouv$ le nombre d'ouvriers au début de la semaine $s$,
  après les embauches et les licenciements.
\end{itemize}

Dans la fonction objectif, il faut maintenant tenir compte du cout de 
ces recrutements et ces licenciements 
ainsi que du coût du salaire des ouvriers, non constant.

En ce qui concerne les contraintes, il faut modifier celles sur 
le nombre de smartphones maximum produits dans l'entreprise. 
Il faut également ajouter les contraintes correspondants 
au nombre d'ouvriers (maximum et minimum).

\subsection*{Fonction objectif}
\begin{align*}
  \mbox{minimiser } 
  \sum_{s=1}^{T} 
  c_m\, \xn &+ (c_m + d_{a,h} \, c_{hs})\, \xsup
  + c_s\, \xstock + c_r\, \xretard + c_{sst}\, \xsst \\
  &+ 35 \, c_{h} \, \nouv 
  + c_{emb} \, \nemb + c_{lic} \, \nlic
\end{align*}

Le tableau~\ref{tab:constantesQuestion7} contient les nouvelles abréviations
des constantes utilisées.
\begin{table}[h]
  \begin{center}
  \begin{tabular}{|l|l|}
    \hline
    Paramètre & Constante représentée \\
    \hline
    \hline
    $c_{h}$ & \texttt{cout\_horaire} \\
    \hline
    $c_{emb}$ & \texttt{cout\_embauche} \\
    \hline
    $c_{lic}$ & \texttt{cout\_licenciement} \\
    \hline
  \end{tabular}
  \caption{Constantes de la modélisation de la ligne d'assemblage
  avec gestion du personnel.}
  \label{tab:constantesQuestion7}
  \end{center}
\end{table}

\subsection*{Contraintes}
Voici les contraintes du problème de la planification 
de la ligne d’assemblage à personnel variable.
\begin{align*}
  \Delta\xstock + \texttts{demande}(s) &= \xn + \xsup 
  + \xretard + \xsst - \myX{s-1}{retard} &\forall s \\
  \Delta\nouv &= \nemb - \nlic &\forall s \\
  \myX{0}{n}&= 0 \\
  \myX{0}{sup}&= 0 \\
  \myX{0}{stock}&= \texttts{stock\_initial} \\
  \myX{0}{retard}&= 0 \\
  \myX{0}{sst}&= 0 \\
  \myN{0}{ouv}&= \texttts{nb\_ouvriers} \\
  \myN{0}{emb}&= 0 \\
  \myN{0}{lic}&= 0 \\
  \myX{T}{stock}&= \texttts{stock\_initial} \\
  \myX{T}{retard}&= 0 \\
  \myX{s-1}{retard} + \Delta\xstock &\leq \xn + \xsup + \xsst &\forall s \\
  \xn &\leq 35\cdot \nouv / d_{a,h}
  &\forall s \\
  \xsup &\leq \texttts{nb\_max\_heure\_sup}\cdot \nouv / d_{a,h}
  &\forall s \\
  \xsst &\leq \texttts{nb\_max\_sous\_traitant} &\forall s \\
  \nouv &\leq \texttts{nb\_max\_ouvriers}  &\forall s \\
  x_s, n_s&\geq 0 &\forall s
\end{align*}
