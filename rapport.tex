\documentclass[12pt,oneside,a4paper]{article}

\usepackage{custom}

\newcommand{\question}
{
\addtocounter{section}{1}
\section*{Question \thesection}
}

\newcommand{\myX}[2]{x_{#1,\text{#2}}}
\newcommand{\xSemaine}[1]{\myX{s}{#1}}
\newcommand{\xn}{\xSemaine{n}}
\newcommand{\xsup}{\xSemaine{sup}}
\newcommand{\xstock}{\xSemaine{stock}}
\newcommand{\xretard}{\xSemaine{retard}}
\newcommand{\xsst}{\xSemaine{sst}}

\newcommand{\texttts}[1]{{\small\texttt{#1}}}

\title{Projet d'optimisation}
\author{Groupe 1}
\date{\today}

\begin{document}

\maketitle

\question %1
\emph{Donnez une formulation linéaire (continue, sans variables entières) 
du problème de la planification de la ligne d'assemblage à personnel constant.
Décrivez successivement variables, contraintes et fonction objectif. 
A ce stade, le fait de ne pas imposer l'intégralité des variables 
vous parait-il problématique ?}

\subsection*{Variables}
Le tableau~\ref{tab:variablesQuestion1} contient les différentes variables $x_{s,\lambda}$
qui correspondent au nombre de smartphones pour chaque semaine $s$
avec la caractéristique $\lambda$.

\begin{table}[h]
  \begin{center}
  \begin{tabular}{|c|l|}
    \hline
    Variable & Caractéristiques des smartphones \\
    \hline
    \hline
    $\xn$ & Produits au \emph{salaire normal}. \\
    \hline
    $\xsup$ & Produits pendant les \emph{heures supplémentaires}. \\
    \hline
    $\xstock$ & Conservés en \emph{stock}. \\
    \hline
    $\xretard$ & Vendus une semaine en \emph{retard}. \\
    \hline
    $\xsst$ & Sous-traités. \\
    \hline
  \end{tabular}
  \caption{Variables de la modélisation de la ligne d'assemblage.}
  \label{tab:variablesQuestion1}
  \end{center}
\end{table}

\subsection*{Contraintes}
Voici les contraintes du problème de la planification 
de la ligne d’assemblage à personnel constant.
On pose que $\Delta x_{s,\lambda} = x_{s,\lambda} - x_{s-1,\lambda}$.
\begin{align*}
  \Delta\xstock + \texttts{demande}(s) &= \xn + \xsup 
  + \xretard + \xsst - \myX{s-1}{retard} &\forall s \\
  \myX{0}{stock} &= \texttts{stock-initial} \\
  \myX{T}{stock} &= \texttts{stock-initial} \\
  \myX{T}{retard}&= 0 \\
  \myX{s-1}{retard} + \Delta\xstock &\leq \xn + \xsup + \xsst &\forall s \\
  \xn &\leq 35\cdot \texttts{nb\_ouvriers}/ d_{a,h}
  &\forall s \\
  \xsup &\leq \texttts{nb\_max\_heure\_sup}\cdot\texttts{nb\_ouvriers}/ d_{a,h}
  &\forall s \\
  \xsst &\leq \texttts{nb\_max\_sous\_traitant} &\forall s \\
  x_s &\geq 0 &\forall s
\end{align*}

\subsection*{Fonction objectif}
\[
  \mbox{minimiser } 
  \sum_{s=1}^{T} 
  c_m\, \xn + (c_m + d_{a,h} \, c_{hs})\, \xsup
  + c_s\, \xstock + c_r\, \xretard + c_{sst}\, \xsst
\]
Le tableau~\ref{tab:constantesQuestion1} contient les abréviations
des constantes utilisées.
\begin{table}[h]
  \begin{center}
  \begin{tabular}{|c|l|}
    \hline
    Paramètre & Constante représentée \\
    \hline
    \hline
    $c_m$ & \texttt{cout\_materiaux} \\
    \hline
    $c_{hs}$ & \texttt{cout\_heure\_sup} \\
    \hline
    $c_s$ & \texttt{cout\_stockage} \\
    \hline
    $c_r$ & \texttt{cout\_retard} \\
    \hline
    $c_{sst}$ & \texttt{cout\_sous\_traitant} \\
    \hline
    $d_{a,h}$ & \texttt{duree\_assemblage}/60 \\
    \hline
  \end{tabular}
  \caption{Constantes de la modélisation de la ligne d'assemblage.}
  \label{tab:constantesQuestion1}
  \end{center}
\end{table}

A ce stade, le fait de ne pas imposer l'intégralité des variables
parait problèmatique dans le sens où les solutions ne sont pas 
garanties d'être entières. Ce qui n'est pas envisageable 
vu que celles-ci représentent des quantités de smartphones.
Par exemple, 
$\xn$ ne sera probablement pas entier si $1/d_{a,h}$ ne l'est pas.
% TODO autre mot que problablement ? "il est possible que" 

\question %2
\emph{Démontrez que, sous certaines hypothèses raisonnables, 
il est possible de garantir que votre modèle linéaire continu admette toujours
une solution entière, c'est-à-dire ne comportant que des quantités produites
entières chaque semaine. 
L'une de ces hypothèses est l'intégralité de la demande chaque semaine ; 
quelles sont les autres ?}

Il est possible de garantir que notre modèle linéaire continu
admette toujours une solution entière sous certaines hypothèses.
Une première hypothèse est que tous les éléments du 
vecteur \texttt{demande} soient entiers.
Il faut également que les constantes \texttt{stock-initial},
\texttt{nb\_max\_heure\_sup}, \texttt{nb\_max\_sous\_traitant},
\texttt{nb\_ouvriers} et $1/d_{a,h}$ soient entières.
% TODO nécéssaire de spécifier même celles qui parraissent évidente? 

\subsubsection*{Preuve}
Pour le prouver, nous allons reformuler notre problème sous la forme d'un problème de flot de coût minimum.
Soit le graphe orienté $G(V,E)$, 
où $V$ représente l'ensemble des noeuds et $E$ l'ensemble des arrêtes.
Il est utile à ce stade de s'aider d'un schéma représentant le graphe. 
Celui-ci est repris à la figure~\ref{fig:schemaFlot}.
$V$ compte un noeud pour chaque semaine et un noeud initial.
On a donc $V := {0, 1, 2, ..., T}$
où $0$ est le noeud initial et $s$ est le noeud de la semaine $s$.
Définissons maintenant les arrêtes de notre graphe.
Pour le noeud $0$, on définit 
\[ V^{-}(0) = \{s_1, s_2, s_3 \, \forall s \ne 0\} \]
avec
\[ \{(0, s_1), (0, s_2), (0, s_3)\} \in E \, \forall s. \]
$s_1$, $s_2$, $s_3$ représentent les différentes manières de produire les smartphones, c'est-à-dire les ouvriers au salaire normal, les ouvriers au salaire des heures supplémentaires et la sous-traitance.
Il y a donc trois arcs entre les noeuds $0$ et $s$.
Pour le noeud initial,
\[ V^{+}(0) = \emptyset \]
Définissons ensuite les arrêtes des noeuds corrspondnats aux semaines
\[ V^{-}(s) = \{s+1\} \qquad \forall s \ne 0 \]
avec 
\[ \{(s, s+1)\} \in E \qquad \forall s \ne 0, T \]
Et,
\[ V^{+}(s) = \{s-1\} \qquad \forall s \ne 0 \]
avec
\[ \{(s, s-1)\} \in E \qquad \forall s \ne 0, 1 \]
Nous devons encore définir les termes sources pour chaques noeuds ainsi que les capacités maximales pour chaque arc.
Soit
\[ b_s = - \texttts{demande}(s) \qquad s \in V \backslash \{0\} \]
Et 
\[ b_0 = \sum_{s=1}^{T} \texttts{demande}(s) \]  
On a aussi
\[ b_1 = \texttts{stock\_initial} \]
Et
\[ b_T = - \texttts{stock\_initial} \]
Soient $h_{ij}$ avec $(i, j) \in E$ les capacités maximales dans l'arc $(i, j)$.
On a $\, \forall s \ne 0$ 
\begin{align*}
  h_{0, s_1} &= 35\cdot \texttts{nb\_ouvriers}/ d_{a,h} \\
  h_{0, s_2} &= \texttts{nb\_max\_heure\_sup}\cdot\texttts{nb\_ouvriers}/ d_{a,h} \\
  h_{0, s_3} &= \texttts{nb\_max\_sous\_traitant} 
\end{align*}
Le graphe maintenant défini, 
on peut définir le problème de minimisation suivant :
\paragraph{Variables}
Soit $x_{ij}$ le flot dans l'arc $(i, j)$.
\paragraph{Objectif}
Le coût total est minimisé.
\[ \sum_{(i, j) \in E} c_{ij} x_{ij} \]
\paragraph{Equations} Le flot est conservé en chaque noeud
\[ \sum_{k \in V^{+}(i)} x_{ik} - \sum_{k \in V^{-}(i)} x_{ki} 
  = b_i \qquad i \in V
\]
Les capacités maximales ne sont pas dépassées
\[ 0 \leq x_{ij} \leq h_{ij} \qquad (i, j) \in E \]

On peut maintenant utiliser le théorème suivant pour conclure que, si nos hypothèses sont vérifiées, notre problème admettra au moins une solution entière.
\paragraph{Théorème}
Si les demandes $b_i$ et les capcités $h_{ij}$ d'un problème de flot de coût minimum sont entières alors il existe une solution optimale entière.

\begin{figure}[hp]
	\centering
		\includegraphics[scale = 0.5]{Schema_flot.png}
	\caption{Schéma représentant le graphe utilisé pour définir 
  le problème de flot de coût minimal.}
	\label{fig:schemaFlot}
\end{figure}

\question %3
\emph{Implémentez sous MATLAB ce modèle linéaire continu, 
et calculez la solution correspondant aux données fournies sur icampus 
(utilisez la fonction \texttt{linprog}). 
Commentez l'allure de la solution obtenue.}

Notre implémentation se trouve dans le fichier \texttt{question3.m}.

\question %4
\emph{Décrivez une procédure permettant, avec le moins de nouveaux calculs
possibles, d'évaluer les conséquences sur la fonction objectif d’une petite
variation de la demande prévue. Plus précisément, analysez l'effet du
remplacement du vecteur \texttt{demande} 
par le vecteur \texttt{demande + epsilon * delta\textunderscore demande} 
où \texttt{delta\textunderscore demande} est un vecteur de perturbation 
sur la demande, 
et \texttt{epsilon} est un paramètre scalaire dont la valeur est faible.}

La perturbation d'une contrainte 
Le problème dual nous donne de telles informations. 
En effet, soit notre problème reformulé sous forme standard : \\
\begin{align*} 
	\text{minimiser} c^Tx &\\
	Ax &= b  \\
	x &\geq 0 
\end{align*}
Une partie du vecteur b correspond exactement à la demande.
Or, si $y_{*}$ est solution du dual de ce problème, la variation de la fonction objectif à une modification $\Delta b$ du vecteur $b$ sera exactement égal à $y_{*}^{T}\Delta b$. \\
En particulier donc, si $\Delta b$ est égal à $$epsilon * delta\_demande$$ aux composantes de $b$ correspondant à la demande et 0 partout ailleurs, $y_{*}^{T}\Delta b$ nous donnera la variation de l'objectif causée par cette variation de la demande.
%Besoin d'une 'preuve' mathématique ? Sachant que ce sera bcp du recopiage des slides...

\question %5

\question %6
Si le coût des heures supplémentaires augmentent, la fonction objectif devient 
\[
  \mbox{minimiser } 
  \sum_{s=1}^{T} 
  c_m\, \xn + (c_m + d_{a,h} \, c_{hs, n})\, \xsup
  + c_s\, \xstock + c_r\, \xretard + c_{sst}\, \xsst
\]
avec $c_{hs, n} = c_{hs, 0} \, \sum_{i=0}^{\xsup \, d_{a, h}} (0.05)^{i} \, {\xsup \, d_{a,h} -1 \choose i}$. Ce modèle n'est plus linéaire.

\end{document}