\documentclass[12pt,oneside,a4paper]{article}

\usepackage{custom}

\newcommand{\question}
{
\addtocounter{section}{1}
\section*{Question \thesection}
}

\newcommand{\myX}[2]{x_{#1,\text{#2}}}
\newcommand{\xSemaine}[1]{\myX{s}{#1}}
\newcommand{\xn}{\xSemaine{n}}
\newcommand{\xsup}{\xSemaine{sup}}
\newcommand{\xstock}{\xSemaine{stock}}
\newcommand{\xretard}{\xSemaine{retard}}
\newcommand{\xsst}{\xSemaine{sst}}

\newcommand{\texttts}[1]{{\small\texttt{#1}}}

\title{Projet d'optimisation}
\author{Groupe 1}
\date{\today}

\begin{document}

\maketitle

\question %1

\subsection*{Variables}
Le tableau~\ref{tab:variablesQuestion1} contient les différentes variables $x_{s,\lambda}$
qui correspondent au nombre de smartphones pour chaque semaine $s$
avec la caractéristique $\lambda$.

\begin{table}[h]
  \begin{center}
  \begin{tabular}{|c|l|}
    \hline
    Variable & Caractéristiques des smartphones \\
    \hline
    \hline
    $\xn$ & Produits au \emph{salaire normal}. \\
    \hline
    $\xsup$ & Produits pendant les \emph{heures supplémentaires}. \\
    \hline
    $\xstock$ & Conservés en \emph{stock}. \\
    \hline
    $\xretard$ & Vendus une semaine en \emph{retard}. \\
    \hline
    $\xsst$ & Sous-traités. \\
    \hline
  \end{tabular}
  \caption{Variables de la modélisation de la ligne d'assemblage.}
  \label{tab:variablesQuestion1}
  \end{center}
\end{table}

\subsection*{Contraintes}
Voici les contraintes du problème de la planification 
de la ligne d’assemblage à personnel constant.
\begin{align*}
  \Delta\xstock + \texttts{demande}(s) &= \xn + \xsup 
  + \xretard + \xsst - \myX{s-1}{retard} &\forall s \\
  \myX{s-1}{retard} + \Delta\xstock &\leq \xn + \xsup + \xsst &\forall s \\
  \myX{0}{stock} &= \texttts{stock-initial} \\
  \myX{T}{stock} &= \texttts{stock-initial} \\
  \myX{T}{retard}&= 0 \\
  \xn &\leq 35\cdot \texttts{nb\_ouvriers}/ d_{a,h}
  &\forall s \\
  \xsup &\leq \texttts{nb\_max\_heure\_sup}\cdot\texttts{nb\_ouvriers}/ d_{a,h}
  &\forall s \\
  \xsst &\leq \texttts{nb\_max\_sous\_traitant} &\forall s \\
  x_s &\geq 0 &\forall s
\end{align*}

Avec
\begin{align*}
  \Delta x_{s,\lambda} &= x_{s,\lambda} - x_{s-1,\lambda} \\
  d_{a,h} &= \texttts{duree\_assemblage}/60.
\end{align*}

\subsection*{Fonction objectif}
\[
  \mbox{minimiser } 
  \sum_{s=1}^{T} 
  c_m\, \xn + \left(c_m + \frac{d_{a}}{60} \, c_{hs} \right) \xsup
  + c_s\, \xstock + c_r\, \xretard + c_{sst}\, \xsst
\]
Le tableau~\ref{tab:constantesQuestion1} contient les abréviations
des constantes utilisées.
\begin{table}[h]
  \begin{center}
  \begin{tabular}{|c|l|}
    \hline
    Paramètre & Constante représentée \\
    \hline
    \hline
    $c_m$ & \texttt{cout\_materiaux} \\
    \hline
    $c_{hs}$ & \texttt{cout\_heure\_sup} \\
    \hline
    $c_s$ & \texttt{cout\_stockage} \\
    \hline
    $c_r$ & \texttt{cout\_retard} \\
    \hline
    $c_{sst}$ & \texttt{cout\_sous\_traitant} \\
    \hline
    $d_a$ & \texttt{duree\_assemblage} \\
    \hline
  \end{tabular}
  \caption{Constantes de la modélisation de la ligne d'assemblage.}
  \label{tab:constantesQuestion1}
  \end{center}
\end{table}

\question %2



\end{document}